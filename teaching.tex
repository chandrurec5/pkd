I owe it to my teachers and colleagues for all that I have learnt, and I sincerely hope to draw inspiration from them while teaching young minds. 
I believe teaching is not merely repeating, but an opportunity to present ideas in a original and refreshing way, a process which is not only an end in itself but a key stepping stone for research.   \par

Experience:  
1) Teaching assistantship: At the Indian Institute of Science, I was a teaching assistant (TA) for courses such as Linear Algebra and Applications (Aug-Dec 2012, 2013, 2014) and Foundations of Data Sciences (Jan-May 2014).  My duties as a TA involved conducting tutorials, setting quizzes and evaluating scripts.
2) Summer School: I  was a member of the organizing committee for the Undergraduate Summer School  conducted by the Department of CSA, IISc. The main aim of the summer school was to introduce undergraduate students to cutting-edge research in computer science via  talks, demos, and hands-on sessions. I have given expository talks on linear algebra and probability at the UG summer school. 

Basic Courses: My course work, experience as an analog design engineer, and the research training during my PhD and post-doctoral days gives the me confidence that  I can handle most undergraduate level courses in computer science, electrical engineering and basic mathematics.

Advanced Courses: I am also interested in designing advanced undergraduate and graduate level courses related to my field of expertise namely reinforcement learning (RL).
RL algorithms learn from samples obtained via direct interaction with the environment, and are different from supervised or unsupervised machine learning algorithms.  RL lies at the intersection of machine learning (ML), stochastic control, optimization and operations research. My primary research focus would be RL and I would like to design courses that enable students to be up to speed with topics in RL. To this end, I would like to design the following advanced undergraduate and graduate courses in ML and RL.

Other activities: Having been a student at the National Institute of Technology, Trichy and at the Indian Institute of Science, I believe that a lot of learning happens when students discuss problems in groups. Hence, I would be interested in moderating and guiding `Technology' clubs by students. By publishing the activities of the club as blogs/videos, the student community can interact with fellow researchers from across the world.

